\subsection{Déterminer par la méthode de votre choix tous les arbres couvrants du graphe G1 suivant}\label{subsec:Q1}

Le nom des sommets de ce graphe ont été choisis en analysant la méthode statique Graph.example() afin d'avoir une cohérence entre les données manipulées par le programme et ce document.

\begin{center}
    \begin{tikzpicture}
        % Les styles utilisé pour visualiser les salles
        \tikzstyle{salle}=[circle, draw]

        \node[salle] (C) {3};
        \node[salle, above=25mm of C] (A) {0};
        \node[salle, right=25mm of C] (D) {1};
        \node[salle, right=25mm of A] (B) {2};

        \draw (A) to (B);
        \draw (A) to (C);
        \draw (A) to (D);
        \draw (B) to (D);
        \draw (C) to (D);
    \end{tikzpicture}

    \textbf{G1}

\end{center}

\newpage
\begin{multicols}{0}

    \subsubsection{Arbre couvrant A}
    \begin{tikzpicture}
        % Les styles utilisé pour visualiser les salles
        \tikzstyle{salle}=[circle, draw]

        \node[salle] (C) {3};
        \node[salle, above=15mm of C] (A) {0};
        \node[salle, right=15mm of C] (D) {1};
        \node[salle, right=15mm of A] (B) {2};

        \draw (A) to (B);
        \draw (B) to (D);
        \draw (C) to (D);
    \end{tikzpicture}

    \subsubsection{Arbre couvrant B}
    \begin{tikzpicture}
        % Les styles utilisé pour visualiser les salles
        \tikzstyle{salle}=[circle, draw]

        \node[salle] (C) {3};
        \node[salle, above=15mm of C] (A) {0};
        \node[salle, right=15mm of C] (D) {1};
        \node[salle, right=15mm of A] (B) {2};

        \draw (A) to (B);
        \draw (B) to (D);
        \draw (A) to (C);
    \end{tikzpicture}

    \subsubsection{Arbre couvrant C}
    \begin{tikzpicture}
        % Les styles utilisé pour visualiser les salles
        \tikzstyle{salle}=[circle, draw]

        \node[salle] (C) {3};
        \node[salle, above=15mm of C] (A) {0};
        \node[salle, right=15mm of C] (D) {1};
        \node[salle, right=15mm of A] (B) {2};

        \draw (A) to (B);
        \draw (A) to (C);
        \draw (C) to (D);
    \end{tikzpicture}

    \subsubsection{Arbre couvrant D}
    \begin{tikzpicture}
        % Les styles utilisé pour visualiser les salles
        \tikzstyle{salle}=[circle, draw]

        \node[salle] (C) {3};
        \node[salle, above=15mm of C] (A) {0};
        \node[salle, right=15mm of C] (D) {1};
        \node[salle, right=15mm of A] (B) {2};

        \draw (A) to (C);
        \draw (C) to (D);
        \draw (D) to (B);
    \end{tikzpicture}

    \subsubsection{Arbre couvrant E}
    \begin{tikzpicture}
        % Les styles utilisé pour visualiser les salles
        \tikzstyle{salle}=[circle, draw]

        \node[salle] (C) {3};
        \node[salle, above=15mm of C] (A) {0};
        \node[salle, right=15mm of C] (D) {1};
        \node[salle, right=15mm of A] (B) {2};

        \draw (A) to (B);
        \draw (A) to (C);
        \draw (A) to (D);
    \end{tikzpicture}

    \subsubsection{Arbre couvrant F}
    \begin{tikzpicture}
        % Les styles utilisé pour visualiser les salles
        \tikzstyle{salle}=[circle, draw]

        \node[salle] (C) {3};
        \node[salle, above=15mm of C] (A) {0};
        \node[salle, right=15mm of C] (D) {1};
        \node[salle, right=15mm of A] (B) {2};

        \draw (A) to (C);
        \draw (A) to (D);
        \draw (B) to (D);
    \end{tikzpicture}

    \subsubsection{Arbre couvrant G}
    \begin{tikzpicture}
        % Les styles utilisé pour visualiser les salles
        \tikzstyle{salle}=[circle, draw]

        \node[salle] (C) {3};
        \node[salle, above=15mm of C] (A) {0};
        \node[salle, right=15mm of C] (D) {1};
        \node[salle, right=15mm of A] (B) {2};

        \draw (D) to (B);
        \draw (D) to (C);
        \draw (D) to (A);
    \end{tikzpicture}

    \subsubsection{Arbre couvrant H}
    \begin{tikzpicture}
        % Les styles utilisé pour visualiser les salles
        \tikzstyle{salle}=[circle, draw]

        \node[salle] (C) {3};
        \node[salle, above=15mm of C] (A) {0};
        \node[salle, right=15mm of C] (D) {1};
        \node[salle, right=15mm of A] (B) {2};

        \draw (A) to (B);
        \draw (A) to (D);
        \draw (C) to (D);
    \end{tikzpicture}
\end{multicols}