%! Author = Mathieu
%! Date = 04/01/2021

% Preamble
\documentclass[11pt]{article}
\usepackage[french]{babel}
\title{Algorithmique et complexité Projet sur les arbres couvrants}
\author{Paul-Emile Watelot et Mathieu Tabary}
\date{4 janvier 2021}

% Packages
% Symboles
\usepackage{amsmath}

% Graphes
\usepackage{pgf}
\usepackage{tikz}
\usetikzlibrary{positioning}

% Mise en page
\usepackage{multicol}

% Document
\begin{document}
    \maketitle
    \newpage
    \tableofcontents
    \newpage

    \section{Introduction}\label{sec:introduction}
        Ce projet a été réalisé par Paul-Emile Watelot et Mathieu Tabary en première année de Master Informatique de Nancy.
        L'objectif est d'implémenter une collection d'algorithmes cherchant des arbres couvrant pour un graphe donné.
        % Algo de Kruskal : https://fr.wikipedia.org/wiki/Algorithme_de_Kruskal
        % Algo de Aldous-Broder : https://en.wikipedia.org/wiki/Maze_generation_algorithm#Aldous-Broder_algorithm
        % Algo de Wilson : https://fr.wikipedia.org/wiki/Arbre_al%C3%A9atoire#Arbre_couvrant_uniforme
        Puis d'utiliser ces algorithmes pour générer des labyrinthes.
    
    \section{Arbres couvrants}\label{sec:arbres-couvrants}
        
\begin{center}
    \begin{tikzpicture}
        % Les styles utilisé pour visualiser les salles
        \tikzstyle{salle}=[circle, draw]

        \node[salle] (A) {A};
        \node[salle, right=25mm of A] (B) {B};
        \node[salle, below=25mm of A] (C) {C};
        \node[salle, below=25mm of B] (D) {D};

        \draw (A) to (B);
        \draw (A) to (C);
        \draw (A) to (D);
        \draw (B) to (D);
        \draw (C) to (D);
    \end{tikzpicture}

    \textbf{G1}

\end{center}

\newpage
\begin{multicols}{2}

    \subsubsection{Arbre couvrant A}
    \begin{tikzpicture}
        % Les styles utilisé pour visualiser les salles
        \tikzstyle{salle}=[circle, draw]

        \node[salle] (A) {A};
        \node[salle, right=15mm of A] (B) {B};
        \node[salle, below=15mm of A] (C) {C};
        \node[salle, below=15mm of B] (D) {D};

        \draw (A) to (B);
        \draw (B) to (D);
        \draw (C) to (D);
    \end{tikzpicture}

    \subsubsection{Arbre couvrant B}
    \begin{tikzpicture}
        % Les styles utilisé pour visualiser les salles
        \tikzstyle{salle}=[circle, draw]

        \node[salle] (A) {A};
        \node[salle, right=15mm of A] (B) {B};
        \node[salle, below=15mm of A] (C) {C};
        \node[salle, below=15mm of B] (D) {D};

        \draw (A) to (B);
        \draw (B) to (D);
        \draw (A) to (C);
    \end{tikzpicture}

    \subsubsection{Arbre couvrant C}
    \begin{tikzpicture}
        % Les styles utilisé pour visualiser les salles
        \tikzstyle{salle}=[circle, draw]

        \node[salle] (A) {A};
        \node[salle, right=15mm of A] (B) {B};
        \node[salle, below=15mm of A] (C) {C};
        \node[salle, below=15mm of B] (D) {D};

        \draw (A) to (B);
        \draw (A) to (C);
        \draw (C) to (D);
    \end{tikzpicture}

    \subsubsection{Arbre couvrant D}
    \begin{tikzpicture}
        % Les styles utilisé pour visualiser les salles
        \tikzstyle{salle}=[circle, draw]

        \node[salle] (A) {A};
        \node[salle, right=15mm of A] (B) {B};
        \node[salle, below=15mm of A] (C) {C};
        \node[salle, below=15mm of B] (D) {D};

        \draw (A) to (C);
        \draw (C) to (D);
        \draw (D) to (B);
    \end{tikzpicture}

    \subsubsection{Arbre couvrant E}
    \begin{tikzpicture}
        % Les styles utilisé pour visualiser les salles
        \tikzstyle{salle}=[circle, draw]

        \node[salle] (A) {A};
        \node[salle, right=15mm of A] (B) {B};
        \node[salle, below=15mm of A] (C) {C};
        \node[salle, below=15mm of B] (D) {D};

        \draw (A) to (B);
        \draw (A) to (C);
        \draw (A) to (D);
    \end{tikzpicture}

    \subsubsection{Arbre couvrant F}
    \begin{tikzpicture}
        % Les styles utilisé pour visualiser les salles
        \tikzstyle{salle}=[circle, draw]

        \node[salle] (A) {A};
        \node[salle, right=15mm of A] (B) {B};
        \node[salle, below=15mm of A] (C) {C};
        \node[salle, below=15mm of B] (D) {D};

        \draw (A) to (C);
        \draw (A) to (D);
        \draw (B) to (D);
    \end{tikzpicture}

    \subsubsection{Arbre couvrant G}
    \begin{tikzpicture}
        % Les styles utilisé pour visualiser les salles
        \tikzstyle{salle}=[circle, draw]

        \node[salle] (A) {A};
        \node[salle, right=15mm of A] (B) {B};
        \node[salle, below=15mm of A] (C) {C};
        \node[salle, below=15mm of B] (D) {D};

        \draw (D) to (B);
        \draw (D) to (C);
        \draw (D) to (A);
    \end{tikzpicture}

    \subsubsection{Arbre couvrant H}
    \begin{tikzpicture}
        % Les styles utilisé pour visualiser les salles
        \tikzstyle{salle}=[circle, draw]

        \node[salle] (A) {A};
        \node[salle, right=15mm of A] (B) {B};
        \node[salle, below=15mm of A] (C) {C};
        \node[salle, below=15mm of B] (D) {D};

        \draw (A) to (B);
        \draw (A) to (D);
        \draw (C) to (D);
    \end{tikzpicture}
\end{multicols}
        \newpage

        \subsection{Tester l'algorithme un million de fois sur le graphe $G_1$,
        et vérifier expérimentalement que les 8 arbres couvrants n'ont pas tous la même probabilité d'apparaître}\label{subsec:Q3}
            En testant un million de fois l'algorithme de Kruskal sur le graphe $G_1$,
            on observe que les arbres couvrants A, B, C et D ont une probabilité d'apparition de 11\% environ,
            alors que les arbres couvrants E, F, G et H ont une probabilité d'apparition de 13\% environ.

        \subsection{Prouver rigoureusement que les 8 arbres couvrants n'ont pas tous la même probabilité d'apparaître.}\label{subsec:Q4}
            Le test sur un million de copies du graphe G1 se fait par la méthode main de la classe MainQuestion3.
            La différence notable entre les arbres couvrants A, B, C et D, face aux arbres couvrants E, F, G et H est l'arête (2, 3).
            Les sommets 2 et 3 ont un degré de 3, alors que les sommets 0 et 1 ont un degré de 2.
            L'arête (2, 3)
    
        \subsection{Implémenter l'algorithme d'Aldous et Broder.
        Tester l'algorithme un million de fois sur le graphe G1,
        et vérifier expérimentalement que les 8 arbres couvrants ont tous la même probabilité d'apparaître.}\label{subsec:Q5}
            Le test sur un million de copies du graphe G1 se fait par la méthode main de la classe MainQuestion5.
            Le nombre d'occurrences et le pourcentage d'apparition de chaque arbre couvrant est indiqué dans le titre de la fenêtre du graphe et au-dessus du dessin du graphe.
            On peut ainsi constater que chaque arbre couvrant a une probabilité d'environ $12,5\%$ , ou $\frac{1}{8}$ d'apparaître.
    
        \subsection{Implémenter l'algorithme de Wilson.
        Tester l'algorithme un million de fois sur le graphe G1,
        et vérifier expérimentalement que les 8 arbres couvrants ont tous la même probabilité d'apparaître.}\label{subsec:Q6}
            Similairement aux questions précédentes, le test se fait par la méthode main de la classe MainQuestion6.
            Le nombre d'occurrences et le pourcentage d'apparition de chaque arbre couvrant est indiqué dans le titre de la fenêtre du graphe et au-dessus du dessin du graphe.
            On constate que chaque arbre couvrant a une probabilité d'environ $12,5\%$ , ou $\frac{1}{8}$ d'apparaître,
            on peut en déduire que cette implantation de l'algorithme de Wilson respecte la répartition uniforme des arbres couvrants.

    \section{Application ludique : les labyrinthes}\label{sec:labyrinthes}
        \subsection{Écrire un algorithme qui crée un labyrinthe 20x20 obtenu avec la méthode de Kruskal
        et un labyrinthe 20x20 obtenu avec la méthode de Aldous-Broder (ou Wilson).}\label{subsec:Q7.}
            Pour créer un labyrinthe, on commence par créer un graphe en forme de grille à la taille du labyrinthe (ici 20x20).
            Ensuite, on cherche un arbre couvrant avec un des algorithmes écrit précédemment.
            Le labyrinthe est cet arbre couvrant.
            Les sommets du graphe sont les salles du labyrinthe et chaque sommet est relié aux sommets de chaque côtés de lui (au-dessus, au-dessous, à gauche et à droite).
            Si une arête est marquée comme $used$, c'est un couloir pour le labyrinthe, sinon c'est un mur.
            Ainsi, avec l'implémentation des algorithmes précédents, la création de labyrinthe est très simplifiée.
            La méthode main de la classe MainQuestion7 créer trois labyrinthes (un par méthode), et créer le fichier .tex associé.


        \subsection{Pour les deux méthodes (Kurskal ou Aldous-Broder / Wilson),
        comparer le nombre moyen de cul de sac, la distance moyenne de l'entrée à la sortie.}\label{subsec:Q8}
        En exécutant la méthode main de la classe MainQuestion8,
        le terminal affichera (pour la méthode Kruskal puis pour la méthode Wilson)
        le nombre de cul de sac moyen et la distance moyenne entre l'entrée et la sortie.
        On constate que la méthode Kruskal produit plus du double de cul de sac que la méthode Wilson.
        Cependant, on constate aussi que la méthode Wilson est bien plus coûteuse en temps que la méthode Kruskal.

\end{document}
