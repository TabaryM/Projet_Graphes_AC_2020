%! Author = Mathieu
%! Date = 04/01/2021

% Preamble
\documentclass[11pt]{article}
\usepackage[french]{babel}
\title{Algorithmique et complexité Projet sur les arbres couvrants}
\author{Paul-Emile Watelot et Mathieu Tabary}
\date{4 janvier 2021}

% Packages
% Symboles
\usepackage{amsmath}

% Graphes
\usepackage{pgf}
\usepackage{tikz}
\usetikzlibrary{positioning}

% Mise en page
\usepackage{multicol}

% Document
\begin{document}
    \maketitle

    \section{Introduction}\label{sec:introduction}
        Ce projet a été réalisé par Paul-Emile Watelot et Mathieu Tabary en première année de Master Informatique de Nancy.
        L'objectif est d'implémenter une collection d'algorithmes cherchant des arbres couvrant pour un graphe donné.
        % Algo de Kruskal : https://fr.wikipedia.org/wiki/Algorithme_de_Kruskal
        % Algo de Aldous-Broder : https://en.wikipedia.org/wiki/Maze_generation_algorithm#Aldous-Broder_algorithm
        % Algo de Wilson : https://fr.wikipedia.org/wiki/Arbre_al%C3%A9atoire#Arbre_couvrant_uniforme
        Puis d'utiliser ces algorithmes pour générer des labyrinthes.
    
    \section{Arbres couvrants}\label{sec:arbres-couvrants}
        \subsection{Déterminer par la méthode de votre choix tous les arbres couvrants du graphe G1 suivant}\label{subsec:Q1}
        
\begin{center}
    \begin{tikzpicture}
        % Les styles utilisé pour visualiser les salles
        \tikzstyle{salle}=[circle, draw]

        \node[salle] (A) {A};
        \node[salle, right=25mm of A] (B) {B};
        \node[salle, below=25mm of A] (C) {C};
        \node[salle, below=25mm of B] (D) {D};

        \draw (A) to (B);
        \draw (A) to (C);
        \draw (A) to (D);
        \draw (B) to (D);
        \draw (C) to (D);
    \end{tikzpicture}

    \textbf{G1}

\end{center}

\newpage
\begin{multicols}{2}

    \subsubsection{Arbre couvrant A}
    \begin{tikzpicture}
        % Les styles utilisé pour visualiser les salles
        \tikzstyle{salle}=[circle, draw]

        \node[salle] (A) {A};
        \node[salle, right=15mm of A] (B) {B};
        \node[salle, below=15mm of A] (C) {C};
        \node[salle, below=15mm of B] (D) {D};

        \draw (A) to (B);
        \draw (B) to (D);
        \draw (C) to (D);
    \end{tikzpicture}

    \subsubsection{Arbre couvrant B}
    \begin{tikzpicture}
        % Les styles utilisé pour visualiser les salles
        \tikzstyle{salle}=[circle, draw]

        \node[salle] (A) {A};
        \node[salle, right=15mm of A] (B) {B};
        \node[salle, below=15mm of A] (C) {C};
        \node[salle, below=15mm of B] (D) {D};

        \draw (A) to (B);
        \draw (B) to (D);
        \draw (A) to (C);
    \end{tikzpicture}

    \subsubsection{Arbre couvrant C}
    \begin{tikzpicture}
        % Les styles utilisé pour visualiser les salles
        \tikzstyle{salle}=[circle, draw]

        \node[salle] (A) {A};
        \node[salle, right=15mm of A] (B) {B};
        \node[salle, below=15mm of A] (C) {C};
        \node[salle, below=15mm of B] (D) {D};

        \draw (A) to (B);
        \draw (A) to (C);
        \draw (C) to (D);
    \end{tikzpicture}

    \subsubsection{Arbre couvrant D}
    \begin{tikzpicture}
        % Les styles utilisé pour visualiser les salles
        \tikzstyle{salle}=[circle, draw]

        \node[salle] (A) {A};
        \node[salle, right=15mm of A] (B) {B};
        \node[salle, below=15mm of A] (C) {C};
        \node[salle, below=15mm of B] (D) {D};

        \draw (A) to (C);
        \draw (C) to (D);
        \draw (D) to (B);
    \end{tikzpicture}

    \subsubsection{Arbre couvrant E}
    \begin{tikzpicture}
        % Les styles utilisé pour visualiser les salles
        \tikzstyle{salle}=[circle, draw]

        \node[salle] (A) {A};
        \node[salle, right=15mm of A] (B) {B};
        \node[salle, below=15mm of A] (C) {C};
        \node[salle, below=15mm of B] (D) {D};

        \draw (A) to (B);
        \draw (A) to (C);
        \draw (A) to (D);
    \end{tikzpicture}

    \subsubsection{Arbre couvrant F}
    \begin{tikzpicture}
        % Les styles utilisé pour visualiser les salles
        \tikzstyle{salle}=[circle, draw]

        \node[salle] (A) {A};
        \node[salle, right=15mm of A] (B) {B};
        \node[salle, below=15mm of A] (C) {C};
        \node[salle, below=15mm of B] (D) {D};

        \draw (A) to (C);
        \draw (A) to (D);
        \draw (B) to (D);
    \end{tikzpicture}

    \subsubsection{Arbre couvrant G}
    \begin{tikzpicture}
        % Les styles utilisé pour visualiser les salles
        \tikzstyle{salle}=[circle, draw]

        \node[salle] (A) {A};
        \node[salle, right=15mm of A] (B) {B};
        \node[salle, below=15mm of A] (C) {C};
        \node[salle, below=15mm of B] (D) {D};

        \draw (D) to (B);
        \draw (D) to (C);
        \draw (D) to (A);
    \end{tikzpicture}

    \subsubsection{Arbre couvrant H}
    \begin{tikzpicture}
        % Les styles utilisé pour visualiser les salles
        \tikzstyle{salle}=[circle, draw]

        \node[salle] (A) {A};
        \node[salle, right=15mm of A] (B) {B};
        \node[salle, below=15mm of A] (C) {C};
        \node[salle, below=15mm of B] (D) {D};

        \draw (A) to (B);
        \draw (A) to (D);
        \draw (C) to (D);
    \end{tikzpicture}
\end{multicols}
        \newpage

        \subsection{Tester l'algorithme un million de fois sur le graphe $G_1$,
        et vérifier expérimentalement que les 8 arbres couvrants n'ont pas tous la même probabilité d'apparaître}\label{subsec:Q3}
            En testant un million de fois l'algorithme de Kruskal sur le graphe $G_1$,
            on observe que les arbres couvrants A, B, C et D ont une probabilité d'apparition de 11\%,
            alors que les arbres couvrants E, F, G et H ont une probabilité d'apparition de 13\% environ.
    
        \subsection{Prouver rigoureusement que les 8 arbres couvrants n'ont pas tous la même probabilité d'apparaître.}\label{subsec:Q4}
            La différence notable entre les arbres couvrants A, B, C et D, face aux arbres couvrants E, F, G et H est l'arête (0, 1).
\end{document}
