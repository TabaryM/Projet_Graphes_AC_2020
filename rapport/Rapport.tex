%! Author = Mathieu
%! Date = 04/01/2021

% Preamble
\documentclass[a4paper,11pt]{article}
\usepackage[french]{babel}
\title{Algorithmique et complexité Projet sur les arbres couvrants}
\author{Paul-Emile Watelot \and Mathieu Tabary}
\date{4 janvier 2021}

% Packages
% Symboles
\usepackage{amsmath}

% Graphes
\usepackage{pgf}
\usepackage{tikz}
\usetikzlibrary{positioning}

% Mise en page
\usepackage{multicol}
\usepackage{hyperref}

% Document
\begin{document}
    \maketitle
    \newpage
    \tableofcontents
    \newpage

    \section{Introduction}\label{sec:introduction}
        Ce projet a été réalisé par Paul-Emile Watelot et Mathieu Tabary en première année de Master Informatique de Nancy.
        Les programmes et questions ont été écrits et traitées par les deux membres de l'équipe a la suite de discussions par chat vocal et de partage d'écrans.

        L'objectif de ce projet est d'implémenter une collection d'algorithmes cherchant un arbre couvrant pour un graphe donné.
        % Algo de Kruskal : https://fr.wikipedia.org/wiki/Algorithme_de_Kruskal
        % Algo de Aldous-Broder : https://en.wikipedia.org/wiki/Maze_generation_algorithm#Aldous-Broder_algorithm
        % Algo de Wilson : https://fr.wikipedia.org/wiki/Arbre_al%C3%A9atoire#Arbre_couvrant_uniforme
        Puis d'utiliser ces algorithmes pour générer des labyrinthes.
    
    \section{Arbres couvrants}\label{sec:arbres-couvrants}
        \subsection{Déterminer par la méthode de votre choix tous les arbres couvrants du graphe G1 suivant}\label{subsec:Q1}

Le nom des sommets de ce graphe ont été choisis en analysant la méthode statique Graph.example() afin d'avoir une cohérence entre les données manipulées par le programme et ce document.

\begin{center}
    \begin{tikzpicture}
        % Les styles utilisé pour visualiser les salles
        \tikzstyle{salle}=[circle, draw]

        \node[salle] (C) {3};
        \node[salle, above=25mm of C] (A) {0};
        \node[salle, right=25mm of C] (D) {1};
        \node[salle, right=25mm of A] (B) {2};

        \draw (A) to (B);
        \draw (A) to (C);
        \draw (A) to (D);
        \draw (B) to (D);
        \draw (C) to (D);
    \end{tikzpicture}

    \textbf{G1}

\end{center}

\newpage
\begin{multicols}{0}

    \subsubsection{Arbre couvrant A}
    \begin{tikzpicture}
        % Les styles utilisé pour visualiser les salles
        \tikzstyle{salle}=[circle, draw]

        \node[salle] (C) {3};
        \node[salle, above=15mm of C] (A) {0};
        \node[salle, right=15mm of C] (D) {1};
        \node[salle, right=15mm of A] (B) {2};

        \draw (A) to (B);
        \draw (B) to (D);
        \draw (C) to (D);
    \end{tikzpicture}

    \subsubsection{Arbre couvrant B}
    \begin{tikzpicture}
        % Les styles utilisé pour visualiser les salles
        \tikzstyle{salle}=[circle, draw]

        \node[salle] (C) {3};
        \node[salle, above=15mm of C] (A) {0};
        \node[salle, right=15mm of C] (D) {1};
        \node[salle, right=15mm of A] (B) {2};

        \draw (A) to (B);
        \draw (B) to (D);
        \draw (A) to (C);
    \end{tikzpicture}

    \subsubsection{Arbre couvrant C}
    \begin{tikzpicture}
        % Les styles utilisé pour visualiser les salles
        \tikzstyle{salle}=[circle, draw]

        \node[salle] (C) {3};
        \node[salle, above=15mm of C] (A) {0};
        \node[salle, right=15mm of C] (D) {1};
        \node[salle, right=15mm of A] (B) {2};

        \draw (A) to (B);
        \draw (A) to (C);
        \draw (C) to (D);
    \end{tikzpicture}

    \subsubsection{Arbre couvrant D}
    \begin{tikzpicture}
        % Les styles utilisé pour visualiser les salles
        \tikzstyle{salle}=[circle, draw]

        \node[salle] (C) {3};
        \node[salle, above=15mm of C] (A) {0};
        \node[salle, right=15mm of C] (D) {1};
        \node[salle, right=15mm of A] (B) {2};

        \draw (A) to (C);
        \draw (C) to (D);
        \draw (D) to (B);
    \end{tikzpicture}

    \subsubsection{Arbre couvrant E}
    \begin{tikzpicture}
        % Les styles utilisé pour visualiser les salles
        \tikzstyle{salle}=[circle, draw]

        \node[salle] (C) {3};
        \node[salle, above=15mm of C] (A) {0};
        \node[salle, right=15mm of C] (D) {1};
        \node[salle, right=15mm of A] (B) {2};

        \draw (A) to (B);
        \draw (A) to (C);
        \draw (A) to (D);
    \end{tikzpicture}

    \subsubsection{Arbre couvrant F}
    \begin{tikzpicture}
        % Les styles utilisé pour visualiser les salles
        \tikzstyle{salle}=[circle, draw]

        \node[salle] (C) {3};
        \node[salle, above=15mm of C] (A) {0};
        \node[salle, right=15mm of C] (D) {1};
        \node[salle, right=15mm of A] (B) {2};

        \draw (A) to (C);
        \draw (A) to (D);
        \draw (B) to (D);
    \end{tikzpicture}

    \subsubsection{Arbre couvrant G}
    \begin{tikzpicture}
        % Les styles utilisé pour visualiser les salles
        \tikzstyle{salle}=[circle, draw]

        \node[salle] (C) {3};
        \node[salle, above=15mm of C] (A) {0};
        \node[salle, right=15mm of C] (D) {1};
        \node[salle, right=15mm of A] (B) {2};

        \draw (D) to (B);
        \draw (D) to (C);
        \draw (D) to (A);
    \end{tikzpicture}

    \subsubsection{Arbre couvrant H}
    \begin{tikzpicture}
        % Les styles utilisé pour visualiser les salles
        \tikzstyle{salle}=[circle, draw]

        \node[salle] (C) {3};
        \node[salle, above=15mm of C] (A) {0};
        \node[salle, right=15mm of C] (D) {1};
        \node[salle, right=15mm of A] (B) {2};

        \draw (A) to (B);
        \draw (A) to (D);
        \draw (C) to (D);
    \end{tikzpicture}
\end{multicols}
        \newpage

        \subsection{Tester l'algorithme un million de fois sur le graphe $G_1$,
        et vérifier expérimentalement que les 8 arbres couvrants n'ont pas tous la même probabilité d'apparaître}\label{subsec:Q3}
            On peut tester un million de fois cet algorithme sur le graphe $G_1$ en exécutant le programme avec l'argument Q3.
            Le nombre d'occurrences et le pourcentage d'apparition de chaque arbre couvrant est
            indiqué dans le titre de la fenêtre du graphe et au-dessus du dessin du graphe.
            Nous pouvons observer que les arbres couvrants A, B, C et D ont une probabilité d'apparition de 11,7\%,
            alors que les arbres couvrants E, F, G et H ont une probabilité d'apparition de 13,3\%.

        \subsection{Prouver rigoureusement que les 8 arbres couvrants n'ont pas tous la même probabilité d'apparaître.}\label{subsec:Q4}
            La différence notable entre les arbres couvrants A, B, C et D et les arbres couvrants E, F, G et H est l'arête (0, 1).
            La probabilité qu'une arête soit prise en première dans l'arbre couvrant suis une loi uniforme
            (chaque arête a autant de probabilité qu'une autre d'être à une position donnée).\\

            Cependant, on peut voir que les arêtes (0, 2), (2, 1), (1, 3) et (3, 0) apparaissent chacune dans 5 arbres couvrants différents,
            alors que l'arête (0, 1) apparait dans seulement 4 arbres couvrants différents.
            De plus, si la première arête choisis pour être dans l'arbre couvrant est (0, 1) (probabilité $\frac{1}{4}$),
            l'arbre couvrant sera E, F, G ou H (il y a 4 arbres couvrants possibles).
            Mais si la première arête choisis n'est pas (0, 1) (probabilité $\frac{1}{5}$),
            il y a 5 arbres couvrants possibles.
            Par exemple si la première arête choisis est (0, 2), les arbres couvrants possibles sont A, B, C, E et H.\\

            Donc, chaque arête à la même probabilité d'être choisis pour être dans un arbre couvrant
            (être la première arête traitée par l'algorithme),
            mais l'arête (0, 1) ne peut produire que 4 arbres couvrants (contre 5 pour les autres arêtes).
            Donc les 4 arbres couvrants produits avec l'arête (0, 1) ont plus de chance d'être produits que les arbres couvrants n'ayant pas l'arête (0, 1).
    
        \subsection{Implémenter l'algorithme d'Aldous et Broder. Tester l'algorithme un million de fois sur le graphe G1,
        et vérifier expérimentalement que les 8 arbres couvrants ont tous la même probabilité d'apparaître.}\label{subsec:Q5}
            De la même manière que pour l'algorithme de Kruskal, le test se fait en exécutant le programme, mais avec l'argument Q5.
            On peut ainsi constater que tous les arbres couvrants ont une probabilité d'apparition de $12,5\%$ , ou $\frac{1}{8}$.

        \subsection{Implémenter l'algorithme de Wilson. Tester l'algorithme un million de fois sur le graphe G1,
        et vérifier expérimentalement que les 8 arbres couvrants ont tous la même probabilité d'apparaître.}\label{subsec:Q6}
            Ce test s'effectue à l'aide de l'argument Q6.
            On constate que chaque arbre couvrant a une probabilité d'apparition de $12,5\%$ , ou $\frac{1}{8}$.

    \section{Application ludique : les labyrinthes}\label{sec:labyrinthes}
        \subsection{Écrire un algorithme qui crée un labyrinthe 20x20 obtenu avec la méthode de Kruskal
        et un labyrinthe 20x20 obtenu avec la méthode de Aldous-Broder (ou Wilson).}\label{subsec:Q7.}
            Pour créer un labyrinthe, on commence par créer un graphe en forme de grille à la taille du labyrinthe (ici 20x20).
            Ensuite, on cherche un arbre couvrant avec un des algorithmes écrit précédemment.

            Le labyrinthe est défini par cet arbre couvrant.
            Les sommets du graphe sont les salles du labyrinthe et chaque sommet est relié aux sommets de chaque côtés de lui
            (au-dessus, au-dessous, à gauche et à droite).
            Si une arête est marquée comme $used$, c'est un couloir pour le labyrinthe, sinon c'est un mur.
            Ainsi, avec l'implémentation des algorithmes précédents, la création de labyrinthe est très simplifiée.

            En exécutant le programme avec l'argument Q7,
            trois labyrinthes (un par méthode) seront créés et un fichier .tex associé sera créé dans le repertoire out.

        \subsection{Pour les deux méthodes (Kurskal ou Aldous-Broder / Wilson),
        comparer le nombre moyen de cul-de-sac, la distance moyenne de l'entrée à la sortie.}\label{subsec:Q8}
            En exécutant le programme avec l'argument Q8,
            le terminal affichera (pour la méthode Kruskal puis pour la méthode Wilson puis pour la méthode AldousBroder)
            le nombre de cul-de-sac moyen, la distance moyenne entre l'entrée et la sortie et le temps d'exécution.\\

            On constate que la méthode Kruskal produit en moyenne environ 120 cul-de-sac,
            tandis que la méthode Wilson ou Aldous et Broder produit environ 116 cul-de-sac.
            On constate aussi que la méthode Kruskal produit un labyrinthe avec une distance
            entre l'entrée et la sortie de 56 cases en moyenne,
            tandis que la méthode Wilson (ou Aldous et Broder) produit des labyrinthes avec une distance
            entre l'entrée et la sortie de 59 cases en moyenne.\\

            Ces différences, bien que notables sur une telle moyenne nous semblent négligeables.
            En effet, résoudre un labyrinthe avec 3\% de cul-de-sac en plus ou en moins ne change pas le défis.
            De la même manière, un labyrinthe dont la distance entre l'entrée et la sortie est 5\%
            plus élevée n'est pas véritablement un labyrinthe plus difficile.

\end{document}
